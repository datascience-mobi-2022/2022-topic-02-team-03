% Options for packages loaded elsewhere
\PassOptionsToPackage{unicode}{hyperref}
\PassOptionsToPackage{hyphens}{url}
%
\documentclass[
  parskip,
  oneside]{scrreprt}
\usepackage{amsmath,amssymb}
\usepackage{lmodern}
\usepackage{iftex}
\ifPDFTeX
  \usepackage[T1]{fontenc}
  \usepackage[utf8]{inputenc}
  \usepackage{textcomp} % provide euro and other symbols
\else % if luatex or xetex
  \usepackage{unicode-math}
  \defaultfontfeatures{Scale=MatchLowercase}
  \defaultfontfeatures[\rmfamily]{Ligatures=TeX,Scale=1}
\fi
% Use upquote if available, for straight quotes in verbatim environments
\IfFileExists{upquote.sty}{\usepackage{upquote}}{}
\IfFileExists{microtype.sty}{% use microtype if available
  \usepackage[]{microtype}
  \UseMicrotypeSet[protrusion]{basicmath} % disable protrusion for tt fonts
}{}
\makeatletter
\@ifundefined{KOMAClassName}{% if non-KOMA class
  \IfFileExists{parskip.sty}{%
    \usepackage{parskip}
  }{% else
    \setlength{\parindent}{0pt}
    \setlength{\parskip}{6pt plus 2pt minus 1pt}}
}{% if KOMA class
  \KOMAoptions{parskip=half}}
\makeatother
\usepackage{xcolor}
\IfFileExists{xurl.sty}{\usepackage{xurl}}{} % add URL line breaks if available
\IfFileExists{bookmark.sty}{\usepackage{bookmark}}{\usepackage{hyperref}}
\hypersetup{
  hidelinks,
  pdfcreator={LaTeX via pandoc}}
\urlstyle{same} % disable monospaced font for URLs
\usepackage{longtable,booktabs,array}
\usepackage{calc} % for calculating minipage widths
% Correct order of tables after \paragraph or \subparagraph
\usepackage{etoolbox}
\makeatletter
\patchcmd\longtable{\par}{\if@noskipsec\mbox{}\fi\par}{}{}
\makeatother
% Allow footnotes in longtable head/foot
\IfFileExists{footnotehyper.sty}{\usepackage{footnotehyper}}{\usepackage{footnote}}
\makesavenoteenv{longtable}
\usepackage{graphicx}
\makeatletter
\def\maxwidth{\ifdim\Gin@nat@width>\linewidth\linewidth\else\Gin@nat@width\fi}
\def\maxheight{\ifdim\Gin@nat@height>\textheight\textheight\else\Gin@nat@height\fi}
\makeatother
% Scale images if necessary, so that they will not overflow the page
% margins by default, and it is still possible to overwrite the defaults
% using explicit options in \includegraphics[width, height, ...]{}
\setkeys{Gin}{width=\maxwidth,height=\maxheight,keepaspectratio}
% Set default figure placement to htbp
\makeatletter
\def\fps@figure{htbp}
\makeatother
\setlength{\emergencystretch}{3em} % prevent overfull lines
\providecommand{\tightlist}{%
  \setlength{\itemsep}{0pt}\setlength{\parskip}{0pt}}
\setcounter{secnumdepth}{5}
\newlength{\cslhangindent}
\setlength{\cslhangindent}{1.5em}
\newlength{\csllabelwidth}
\setlength{\csllabelwidth}{3em}
\newlength{\cslentryspacingunit} % times entry-spacing
\setlength{\cslentryspacingunit}{\parskip}
\newenvironment{CSLReferences}[2] % #1 hanging-ident, #2 entry spacing
 {% don't indent paragraphs
  \setlength{\parindent}{0pt}
  % turn on hanging indent if param 1 is 1
  \ifodd #1
  \let\oldpar\par
  \def\par{\hangindent=\cslhangindent\oldpar}
  \fi
  % set entry spacing
  \setlength{\parskip}{#2\cslentryspacingunit}
 }%
 {}
\usepackage{calc}
\newcommand{\CSLBlock}[1]{#1\hfill\break}
\newcommand{\CSLLeftMargin}[1]{\parbox[t]{\csllabelwidth}{#1}}
\newcommand{\CSLRightInline}[1]{\parbox[t]{\linewidth - \csllabelwidth}{#1}\break}
\newcommand{\CSLIndent}[1]{\hspace{\cslhangindent}#1}
\usepackage[ngerman, main=english]{babel}
\usepackage[utf8]{inputenc}
\usepackage[T1]{fontenc}
\usepackage{lmodern}
\usepackage[onehalfspacing]{setspace}
\usepackage[left=2.50cm, right=2.50cm, top=2.50cm, bottom=2.50cm, bindingoffset=10mm, includehead, includefoot]{geometry}
\usepackage[headsepline]{scrlayer-scrpage}
\usepackage{url}
\usepackage[backend=biber, style=authoryear, giveninits=true, maxbibnames=99, uniquename=init, maxcitenames=2, hyperref=true, date=year]{biblatex}
\usepackage{setspace}
\usepackage{xpatch}
\usepackage{csquotes}
\usepackage{amsmath}
\usepackage{listings}
\usepackage{booktabs}
\usepackage{longtable}
\usepackage{multirow}
\usepackage{rotating}
\usepackage{subfigure}
\usepackage{graphicx}
\usepackage{float}
\usepackage{acronym}
\usepackage{lipsum}
\usepackage{scrhack}
\emergencystretch=50pt
\clubpenalty = 10000
\widowpenalty = 10000
\displaywidowpenalty = 10000
\automark[section]{chapter}
\renewcommand*{\chaptermarkformat}{}
\renewcommand*{\sectionmarkformat}{}
\setkomafont{title}{\sffamily}
\setkomafont{disposition}{\usekomafont{title}}
\setkomafont{author}{\usekomafont{title}}
\setkomafont{date}{\usekomafont{title}}
\setkomafont{caption}{\sffamily\small}
\setkomafont{captionlabel}{\usekomafont{caption}\bfseries\small}
\setkomafont{pagehead}{\normalfont\scshape}
\ifLuaTeX
  \usepackage{selnolig}  % disable illegal ligatures
\fi

\author{}
\date{\vspace{-2.5em}}

\begin{document}

\begin{titlepage}
\centering
    {\Large Ruprecht-Karls-Universität Heidelberg\\
        Fakultät für Biowissenschaften\\
        Bachelorstudiengang Molekulare Biotechnologie\\}

    {\vspace{\stretch{2}}}
    {\doublelespacing{\usekomafont{title}}

        {\Huge Cancer Hallmark and Metabolic Pathways in Cancer}

      {\Huge Topic 02 Team 03}

        {\Huge Exploration of Lung Adenocarcinoma (LUAD)}

    }

    \vspace{\stretch{2}}
    {\Large Data Science Project SoSe 2022}

    \vspace{\stretch{2}}

    {\Large
        \begin{tabular}{rl}
            Autoren & Paul Brunner, Marie Kleinert, Felipe Stünkel, Chloé Weiler\\
            Abgabetermin &18.07.2022\\
        \end{tabular}
    }

    \vspace{\stretch{1}}

\end{titlepage}

\hypertarget{introduction}{%
\chapter{Introduction}\label{introduction}}

\hypertarget{biological-background}{%
\section{Biological Background}\label{biological-background}}

\hypertarget{principal-component-analysis-pca}{%
\section{Principal Component Analysis
(PCA)}\label{principal-component-analysis-pca}}

Principal component analysis (PCA) is a procedure used to perform linear
dimension reduction. The goal is to reduce the dimension of a given data
set whilst losing as little information as possible by retaining a
maximum of the standardized data set's variation (Ringnér, 2008).

Principal components (PC) are a set of new orthogonal variables that are
made up of a linear combination of the original variables. Principal
components display the pattern of similarity of the observations and of
the variables as points in maps (Abdi and Williams, 2010). By
convention, the PCs are ordered in decreasing ~order according to the
amount of variation they explain of the original data (Ringnér, 2008).
It is important to note that all PCs are uncorrelated.

PCA is a useful tool for genome-wide expression studies and often serves
as a first step before clustering or classification of the data.
Dimension reduction is a necessary step for easy data exploration and
visualization (Ringnér, 2008).

\hypertarget{uniform-manifold-approximation-and-projection-umap}{%
\section{Uniform Manifold Approximation and Projection
(UMAP)}\label{uniform-manifold-approximation-and-projection-umap}}

Uniform manifold approximation and projection (UMAP) is a k-neighbour
graph based algorithm that is used for nonlinear dimension reduction
(Smets et al., 2019) (McInnes et al., 2018).

After data normalization, the Euclidean distances between points in a
two-dimensional space of the graph are calculated and a local radius is
determined (Vermeulen et al., 2021). In general the closer two points
are to each other, the more similar they are. UMAP makes a density
estimation to find the right local radius. This variable radius is
smaller in high density regions of data points and larger in low density
regions. In general, the density is higher when the k-nearest neighbour
is close and vice versa. The number of k-nearest neighbours controls the
number of neighbours whose local topology is preserved. Precisely, a
large number of neighbours will ensure that more global structure is
preserved whereas a smaller number of neighbours will ensure the
preservation of more local structure (McInnes et al., 2018).

UMAP is a newer method tan PCA and it is generally believed to be easier
to interpret and group data than by using PCA. Furthmore, UMAP has the
advantage of not requiring linear data (Milošević et al., 2022).

\hypertarget{gene-set-enrichment-analysis-gsea}{%
\section{Gene Set Enrichment Analysis
(GSEA)}\label{gene-set-enrichment-analysis-gsea}}

Gene set enrichment analysis (GSEA) is a computational method that is
used to determine whether two gene expression states are significantly
different from each other or not. In this project we compared gene
expression profiles between healthy and tumorous tissue of
LUAD(Subramanian et al., 2007).

Two data sets are compared and the genes are sorted from the most to the
least differential expression between the data sets according to their
p-values. This creates a ranked list L.

Referring to an \emph{a priori} defined set of gene sets S, the goal is
to locate for each pathway of S where its corresponding genes fall in L
and find a discerning trend. If the genes of a given pathway are
randomly distributed in L then the pathway is assumed to not
significantly contribute to the particular tumor's phenotype. However if
the genes are primarily clustered at the top or the bottom of L then a
phenotypic significance of the given pathway can be assumed.

To determine the location of the genes, an enrichment score is
calculated for each pathway. For this, a running-sum statistic is
calculated as the list L is ran through. The running-sum is increased
every time a gene belonging to the pathway in question is encountered
and decreased otherwise. An enrichment score is thus calculated for each
pathway. The enrichment score is defined as the maximum deviation from
zero of the running-sum.

Lastly, adjustment for multiple hypothesis testing is performed by
normalizing the enrichment score for each pathway to account for its
size and a normalized enrichment score is obtained.

GSEA is a useful tool for interpretation of gene expression data.

(Subramanian et al., 2005)

\hypertarget{gene-set-variation-analysis-gsva}{%
\section{Gene Set Variation Analysis
(GSVA)}\label{gene-set-variation-analysis-gsva}}

Gene set variation analysis (GSVA) is an unsupervised method to estimate
pathway activities based on gene expression data. Contrarily to the
aforementioned GSEA, GSVA does not rely on phenotypic characterisation
of the data sets into two categories but rather quantifies enrichment in
a sample-wise manner which makes GSVA the better choice to perform on
the tcga\_exp data set.

GSVA estimates a cumulative distribution for each gene over all samples.
The gene expression values are then converted according to these
estimated cumulative distributions into scaled values. Based on these
new values, the genes are ranked in each sample. Next, the genes are
classified into two distributions and a Komogorow-Smirnow statistic is
calculated to judge how similar the distributions are to each other and
to obtain an ES.

The GSVA corresponds to either the maximum deviation between both
running sums or the GSVA score can be defined as the difference of the
maximum deviations in the positive and in the negative direction. A
highly positive or negative GSVA score indicates that the studied gene
set is positively or negatively enriched compared to the genes not in
the gene set, respectively. If the GSVA score for a given gene set is
close to zero, then the gene set is probably not differentially
expressed compared to the genes not in the gene set.

(Hänzelmann et al., 2013)

\hypertarget{abbreviations}{%
\chapter{Abbreviations}\label{abbreviations}}

\begin{longtable}[]{@{}ll@{}}
\toprule
\endhead
GSEA & gene set enrichment analysis \\
GSVA & gene set variation analysis \\
LUAD & lung adenocarcinoma \\
PC & principal component \\
PCA & principal component analysis \\
TCGA & The cancer genome atlas \\
UMAP & uniform manifold approximation and projection \\
\bottomrule
\end{longtable}

\hypertarget{methods}{%
\chapter{Methods}\label{methods}}

\hypertarget{overview-of-used-packages}{%
\section{Overview of used packages}\label{overview-of-used-packages}}

\begin{longtable}[]{@{}
  >{\raggedright\arraybackslash}p{(\columnwidth - 4\tabcolsep) * \real{0.1364}}
  >{\raggedright\arraybackslash}p{(\columnwidth - 4\tabcolsep) * \real{0.7045}}
  >{\raggedright\arraybackslash}p{(\columnwidth - 4\tabcolsep) * \real{0.1591}}@{}}
\caption{\textbf{Tab. 1: Used packages in alphabetical
order.}}\tabularnewline
\toprule
\begin{minipage}[b]{\linewidth}\raggedright
Package Name
\end{minipage} & \begin{minipage}[b]{\linewidth}\raggedright
Application
\end{minipage} & \begin{minipage}[b]{\linewidth}\raggedright
\textbf{Reference}
\end{minipage} \\
\midrule
\endfirsthead
\toprule
\begin{minipage}[b]{\linewidth}\raggedright
Package Name
\end{minipage} & \begin{minipage}[b]{\linewidth}\raggedright
Application
\end{minipage} & \begin{minipage}[b]{\linewidth}\raggedright
\textbf{Reference}
\end{minipage} \\
\midrule
\endhead
babyplots & create interactive 3D visualizations & Trost (2022) \\
base & basic R functions & R Core Team (2022a) \\
bayesbio & calculate Jaccard coefficients & McKenzie (2016) \\
BiocParallel & novel implementations of functions for parallel
evaluation & (Morgan et al., 2021) \\
biomaRt & access to genome databases & Durinck et al. (2009) \\
cinaR & combination of different packages & Karakaslar and Ucar
(2022) \\
cluster & cluster analysis of data & Maechler et al. (2021) \\
ComplexHeatmap & arrange multiple heatmaps & (Gu et al., 2016) \\
edgeR & assess differential expression in gene expression profiles &
Chen et al. (2016) \\
EnhancedVolcano & produce improved volcano plots & (Blighe et al.,
2021) \\
enrichplot & visualization of gene set enrichment results (GSEA) & Yu
(2022) \\
FactoMineR & perform principal component analysis (PCA) & Lê et al.
(2008) \\
fgsea & Run GSEA on a pre-ranked list & Korotkevich et al. (2019) \\
ggplot2 & visualization of results in dot plots, bar plots and box plots
& Wickham (2016) \\
ggpubr & formatting of ggplot2-based graphs & Kassambara (2020) \\
grid & implements the primitive graphical functions that underlie the
ggplot2 plotting system & R Core Team (2022b) \\
gridExtra & arrange multiple plots on a page & Auguie (2017) \\
GSVA & Run GSVA on a data set & (\textbf{GSVA?}) \\
gplots & plotting data & (Warnes et al., 2022a) \\
gtools & calculate foldchange, find NAs,

logratio2foldchange & Warnes et al. (2022b) \\
knitr & creation of citations using write\_bib & Xie (2014) \\
limma & ``linear models for microarray data'' & Ritchie et al. (2015) \\
msigdbr & provides the `Molecular Signatures Database' (MSigDB) gene
sets & Dolgalev (2022) \\
parallel & allows for parallel computation through multi core processing
& R Core Team (2022c) \\
pheatmap & draw clustered heatmaps & Kolde (2019) \\
RColorBrewer & provides color schemes for maps & Neuwirth (2022) \\
scales & helps in visualization: r automatically determines breaks and
labels for axes and legends & Wickham and Seidel (2022) \\
Seurat & visualize gene set enrichment results in dot plots & Satija et
al. (2022) \\
tidyverse & collection of R packages, including ggplot2 & Wickham et al.
(2019) \\
uwot & performs dimensionality reduction and Uniform Manifold
Approxiamtion and Projection (UMAP) & Melville (2021) \\
\bottomrule
\end{longtable}

\hypertarget{our-data}{%
\section{Our Data}\label{our-data}}

\tableofcontents

\hypertarget{result}{%
\chapter{Result}\label{result}}

\hypertarget{discussion}{%
\chapter{Discussion}\label{discussion}}

\hypertarget{outlook}{%
\chapter{Outlook}\label{outlook}}

\hypertarget{references}{%
\chapter{References}\label{references}}

\hypertarget{refs}{}
\begin{CSLReferences}{0}{0}
\leavevmode\vadjust pre{\hypertarget{ref-abdi2010}{}}%
Abdi, H., and Williams, L.J. (2010).
\href{https://doi.org/10.1002/wics.101}{Principal component analysis}.
WIREs Computational Statistics \emph{2}, 433--459.

\leavevmode\vadjust pre{\hypertarget{ref-gridExtra}{}}%
Auguie, B. (2017).
\href{https://CRAN.R-project.org/package=gridExtra}{gridExtra:
Miscellaneous functions for "grid" graphics}.

\leavevmode\vadjust pre{\hypertarget{ref-enhancedvolcano}{}}%
Blighe, K., Rana, S., and Lewis, M. (2021).
\href{https://github.com/kevinblighe/EnhancedVolcano}{EnhancedVolcano:
Publication-ready volcano plots with enhanced colouring and labeling}.

\leavevmode\vadjust pre{\hypertarget{ref-edgeR}{}}%
Chen, Y., Lun, A.A.T., and Smyth, G.K. (2016).
\href{https://doi.org/10.12688/f1000research.8987.2}{From reads to genes
to pathways: Differential expression analysis of RNA-seq experiments
using rsubread and the edgeR quasi-likelihood pipeline}. F1000Research
\emph{5}, 1438.

\leavevmode\vadjust pre{\hypertarget{ref-msigdbr}{}}%
Dolgalev, I. (2022).
\href{https://CRAN.R-project.org/package=msigdbr}{Msigdbr: MSigDB gene
sets for multiple organisms in a tidy data format}.

\leavevmode\vadjust pre{\hypertarget{ref-biomaRt2009}{}}%
Durinck, S., Spellman, P.T., Birney, E., and Huber, W. (2009). Mapping
identifiers for the integration of genomic datasets with the
r/bioconductor package biomaRt. Nature Protocols \emph{4}, 1184--1191.

\leavevmode\vadjust pre{\hypertarget{ref-complexheatmap}{}}%
Gu, Z., Eils, R., and Schlesner, M. (2016). Complex heatmaps reveal
patterns and correlations in multidimensional genomic data.
Bioinformatics.

\leavevmode\vadjust pre{\hypertarget{ref-hanzelmann2013}{}}%
Hänzelmann, S., Castelo, R., and Guinney, J. (2013). GSVA: Gene set
variation analysis for microarray and RNA-seq data. BMC Bioinformatics
\emph{14}, 1--15.

\leavevmode\vadjust pre{\hypertarget{ref-cinaR}{}}%
Karakaslar, O., and Ucar, D. (2022).
\href{https://CRAN.R-project.org/package=cinaR}{cinaR: A computational
pipeline for bulk 'ATAC-seq' profiles}.

\leavevmode\vadjust pre{\hypertarget{ref-ggpubr}{}}%
Kassambara, A. (2020).
\href{https://CRAN.R-project.org/package=ggpubr}{Ggpubr: 'ggplot2' based
publication ready plots}.

\leavevmode\vadjust pre{\hypertarget{ref-pheatmap}{}}%
Kolde, R. (2019).
\href{https://CRAN.R-project.org/package=pheatmap}{Pheatmap: Pretty
heatmaps}.

\leavevmode\vadjust pre{\hypertarget{ref-fgsea}{}}%
Korotkevich, G., Sukhov, V., and Sergushichev, A. (2019).
\href{https://doi.org/10.1101/060012}{Fast gene set enrichment
analysis}. bioRxiv.

\leavevmode\vadjust pre{\hypertarget{ref-FactoMineR}{}}%
Lê, S., Josse, J., and Husson, F. (2008).
\href{https://doi.org/10.18637/jss.v025.i01}{{FactoMineR}: A package for
multivariate analysis}. Journal of Statistical Software \emph{25},
1--18.

\leavevmode\vadjust pre{\hypertarget{ref-cluster}{}}%
Maechler, M., Rousseeuw, P., Struyf, A., Hubert, M., and Hornik, K.
(2021). \href{https://CRAN.R-project.org/package=cluster}{Cluster:
Cluster analysis basics and extensions}.

\leavevmode\vadjust pre{\hypertarget{ref-mcinnes2018}{}}%
McInnes, L., Healy, J., and Melville, J. (2018).
\href{https://doi.org/10.48550/ARXIV.1802.03426}{UMAP: Uniform manifold
approximation and projection for dimension reduction} (arXiv).

\leavevmode\vadjust pre{\hypertarget{ref-bayesbio}{}}%
McKenzie, A. (2016).
\href{https://CRAN.R-project.org/package=bayesbio}{Bayesbio:
Miscellaneous functions for bioinformatics and bayesian statistics}.

\leavevmode\vadjust pre{\hypertarget{ref-uwot}{}}%
Melville, J. (2021).
\href{https://CRAN.R-project.org/package=uwot}{Uwot: The uniform
manifold approximation and projection (UMAP) method for dimensionality
reduction}.

\leavevmode\vadjust pre{\hypertarget{ref-milosevic2022}{}}%
Milošević, D., Medeiros, A.S., Stojković Piperac, M., Cvijanović, D.,
Soininen, J., Milosavljević, A., and Predić, B. (2022).
\href{https://doi.org/10.1016/j.scitotenv.2021.152365}{The application
of uniform manifold approximation and projection (UMAP) for
unconstrained ordination and classification of biological indicators in
aquatic ecology}. Science of The Total Environment \emph{815}, 152365.

\leavevmode\vadjust pre{\hypertarget{ref-biocparallel}{}}%
Morgan, M., Wang, J., Obenchain, V., Lang, M., Thompson, R., and Turaga,
N. (2021).
\href{https://github.com/Bioconductor/BiocParallel}{BiocParallel:
Bioconductor facilities for parallel evaluation}.

\leavevmode\vadjust pre{\hypertarget{ref-RColorBrewer}{}}%
Neuwirth, E. (2022).
\href{https://CRAN.R-project.org/package=RColorBrewer}{RColorBrewer:
ColorBrewer palettes}.

\leavevmode\vadjust pre{\hypertarget{ref-R-base}{}}%
R Core Team (2022a). \href{https://www.R-project.org/}{R: A language and
environment for statistical computing} (Vienna, Austria: R Foundation
for Statistical Computing).

\leavevmode\vadjust pre{\hypertarget{ref-grid}{}}%
R Core Team (2022b). \href{https://www.R-project.org/}{R: A language and
environment for statistical computing} (Vienna, Austria: R Foundation
for Statistical Computing).

\leavevmode\vadjust pre{\hypertarget{ref-parallel}{}}%
R Core Team (2022c). \href{https://www.R-project.org/}{R: A language and
environment for statistical computing} (Vienna, Austria: R Foundation
for Statistical Computing).

\leavevmode\vadjust pre{\hypertarget{ref-ringner2008}{}}%
Ringnér, M. (2008). What is principal component analysis? Nature
Biotechnology \emph{26}, 303--304.

\leavevmode\vadjust pre{\hypertarget{ref-limma2015}{}}%
Ritchie, M.E., Phipson, B., Wu, D., Hu, Y., Law, C.W., Shi, W., and
Smyth, G.K. (2015). \href{https://doi.org/10.1093/nar/gkv007}{{limma}
powers differential expression analyses for {RNA}-sequencing and
microarray studies}. Nucleic Acids Research \emph{43}, e47.

\leavevmode\vadjust pre{\hypertarget{ref-R-SeuratObject}{}}%
Satija, R., Butler, A., Hoffman, P., and Stuart, T. (2022).
\href{https://CRAN.R-project.org/package=SeuratObject}{SeuratObject:
Data structures for single cell data}.

\leavevmode\vadjust pre{\hypertarget{ref-smets2019}{}}%
Smets, T., Verbeeck, N., Claesen, M., Asperger, A., Griffioen, G.,
Tousseyn, T., Waelput, W., Waelkens, E., and De Moor, B. (2019).
\href{https://doi.org/10.1021/acs.analchem.8b05827}{Evaluation of
distance metrics and spatial autocorrelation in uniform manifold
approximation and projection applied to mass spectrometry imaging data}.
Analytical Chemistry \emph{91}, 5706--5714.

\leavevmode\vadjust pre{\hypertarget{ref-subramanian2005}{}}%
Subramanian, A., Tamayo, P., Mootha, V.K., Mukherjee, S., Ebert, B.L.,
Gillette, M.A., Paulovich, A., Pomeroy, S.L., Golub, T.R., Lander, E.S.,
et al. (2005). \href{https://doi.org/10.1073/pnas.0506580102}{Gene set
enrichment analysis: A knowledge-based approach for interpreting
genome-wide expression profiles}. Proceedings of the National Academy of
Sciences \emph{102}, 15545--15550.

\leavevmode\vadjust pre{\hypertarget{ref-subramanian2007}{}}%
Subramanian, A., Kuehn, H., Gould, J., Tamayo, P., and Mesirov, J.P.
(2007). \href{https://doi.org/10.1093/bioinformatics/btm369}{{GSEA-P: a
desktop application for Gene Set Enrichment Analysis}}. Bioinformatics
\emph{23}, 3251--3253.

\leavevmode\vadjust pre{\hypertarget{ref-babyplots}{}}%
Trost, N. (2022). Babyplots: Easy, fast, interactive 3D visualizations
for data exploration and presentation.

\leavevmode\vadjust pre{\hypertarget{ref-VERMEULEN2021119547}{}}%
Vermeulen, M., Smith, K., Eremin, K., Rayner, G., and Walton, M. (2021).
\href{https://doi.org/10.1016/j.saa.2021.119547}{Application of uniform
manifold approximation and projection (UMAP) in spectral imaging of
artworks}. Spectrochimica Acta Part A: Molecular and Biomolecular
Spectroscopy \emph{252}, 119547.

\leavevmode\vadjust pre{\hypertarget{ref-R-gtools}{}}%
Warnes, G.R., Bolker, B., and Lumley, T. (2022b).
\href{https://github.com/r-gregmisc/gtools}{Gtools: Various r
programming tools}.

\leavevmode\vadjust pre{\hypertarget{ref-gplots}{}}%
Warnes, G.R., Bolker, B., Bonebakker, L., Gentleman, R., Huber, W.,
Liaw, A., Lumley, T., Maechler, M., Magnusson, A., Moeller, S., et al.
(2022a). \href{https://CRAN.R-project.org/package=gplots}{Gplots:
Various r programming tools for plotting data}.

\leavevmode\vadjust pre{\hypertarget{ref-ggplot2}{}}%
Wickham, H. (2016). \href{https://ggplot2.tidyverse.org}{ggplot2:
Elegant graphics for data analysis} (Springer-Verlag New York).

\leavevmode\vadjust pre{\hypertarget{ref-scales}{}}%
Wickham, H., and Seidel, D. (2022).
\href{https://CRAN.R-project.org/package=scales}{Scales: Scale functions
for visualization}.

\leavevmode\vadjust pre{\hypertarget{ref-tidyverse}{}}%
Wickham, H., Averick, M., Bryan, J., Chang, W., McGowan, L.D., François,
R., Grolemund, G., Hayes, A., Henry, L., Hester, J., et al. (2019).
\href{https://doi.org/10.21105/joss.01686}{Welcome to the {tidyverse}}.
Journal of Open Source Software \emph{4}, 1686.

\leavevmode\vadjust pre{\hypertarget{ref-knitr2014}{}}%
Xie, Y. (2014).
\href{http://www.crcpress.com/product/isbn/9781466561595}{Knitr: A
comprehensive tool for reproducible research in {R}}. In Implementing
Reproducible Computational Research, V. Stodden, F. Leisch, and R.D.
Peng, eds. (Chapman; Hall/CRC),.

\leavevmode\vadjust pre{\hypertarget{ref-enrichplot}{}}%
Yu, G. (2022).
\href{https://yulab-smu.top/biomedical-knowledge-mining-book/}{Enrichplot:
Visualization of functional enrichment result}.

\end{CSLReferences}

\hypertarget{appendix}{%
\chapter{Appendix}\label{appendix}}

\end{document}
